%%%%%%%%%%%%%%%%%%%%%%%%%%%%%%%%%%%%%%%%%
% Professional Mathematical Presentation Template
% 
% This template uses the beamer class with the Madrid theme
% and a custom color scheme for a clean, professional look
% that works well with mathematical content.
%%%%%%%%%%%%%%%%%%%%%%%%%%%%%%%%%%%%%%%%%

\documentclass[aspectratio=169]{beamer} % 16:9 aspect ratio (modern)

% Theme settings
\usetheme{Madrid}
\usecolortheme{default}


\definecolor{primcolor}{RGB}{25,50,100} % Dark blue
\setbeamercolor{structure}{fg=primcolor}
\setbeamercolor{frametitle}{bg=primcolor!15, fg=primcolor}
\setbeamercolor{title}{fg=white} % White title text for contrast
\setbeamercolor{subtitle}{fg=white} % White subtitle text
\setbeamercolor{author}{fg=primcolor} % White author text
\setbeamercolor{date}{fg=primcolor} % White date text
\setbeamercolor{institute}{fg=primcolor} % White institute text

% Font settings
\usefonttheme{professionalfonts}
\usefonttheme{serif}

% Package imports
\usepackage{amsmath, amsfonts, amssymb, amsthm} % Math packages
\usepackage{mathtools} % Enhanced math tools
\usepackage{bm} % Bold math symbols
\usepackage{graphicx} % For images
\usepackage{booktabs} % Professional tables
\usepackage{tikz} % For diagrams
\usetikzlibrary{arrows, positioning, matrix, decorations.pathreplacing}

% Use beamer's theorem styles
\setbeamertemplate{theorem}[ams style]
\setbeamertemplate{theorems}[numbered]


% Remove navigation symbols
\setbeamertemplate{navigation symbols}{}

% Custom footer
\setbeamertemplate{footline}{
  \leavevmode%
  \hbox{%
  \begin{beamercolorbox}[wd=.333333\paperwidth,ht=2.25ex,dp=1ex,center]{author in head/foot}%
    \usebeamerfont{author in head/foot}\insertshortauthor
  \end{beamercolorbox}%
  \begin{beamercolorbox}[wd=.333333\paperwidth,ht=2.25ex,dp=1ex,center]{title in head/foot}%
    \usebeamerfont{title in head/foot}\insertshorttitle
  \end{beamercolorbox}%
  \begin{beamercolorbox}[wd=.333333\paperwidth,ht=2.25ex,dp=1ex,right]{date in head/foot}%
    \usebeamerfont{date in head/foot}\insertshortdate{}\hspace*{2em}
    \insertframenumber{} / \inserttotalframenumber\hspace*{2ex} 
  \end{beamercolorbox}}%
  \vskip0pt%
}

% Title information
\title[DRL]{Deep Reinforcement Learning can promote sustainable human behaviour in common-pool resource problem}
\subtitle{Koster et al. 2025}
\author[Longye]{Longye Tian \\ \texttt{longye.tian@anu.edu.au}}
\institute[ANU]{Australian National University\\ School of Economics}
\date{March 27th, 2025}
\DeclareFontFamily{U}{mathx}{\hyphenchar\font45}
\DeclareFontShape{U}{mathx}{m}{n}{
      <5> <6> <7> <8> <9> <10>
      <10.95> <12> <14.4> <17.28> <20.74> <24.88>
      mathx10
      }{}
\DeclareSymbolFont{mathx}{U}{mathx}{m}{n}
\DeclareMathSymbol{\bigtimes}{1}{mathx}{"91}

\begin{document}

% Title frame
\begin{frame}
  \titlepage
\end{frame}

% Outline frame
\begin{frame}{Outline}
  \tableofcontents
\end{frame}
\section{Motivation}
\begin{frame}{Motivation}
\begin{itemize}
    \item how to allocate public resources to achieve sustainability?
    \item Agents are not allowed to communicate, contract, etc.
    \item Fina a resource allocation rule that encourage reciprocation
    \item Care about equality, sustainability, and quality of life.
\end{itemize}
\end{frame}

\begin{frame}{Main Idea}
\begin{enumerate}
    \item Collect data from human players under various allocation mechanism
    \item \textcolor{blue}{Train the neural network, behavioral clones to imitate human behaviors}
    \item Behavioral clones create more data for the RL agents to design a mechanism 
    \item Adopt that RL-based mechanism to real human to validate its performace
    \item Interpret the RL-based mechanism.
\end{enumerate}
\end{frame}

\begin{frame}{Key points}
\begin{itemize}
    \item Behavioral Clone is something I really like
    \item This is an more applied paper 
\end{itemize}
    
\end{frame}



\begin{frame}{Discussion}
\begin{itemize}
    \item the use of behaviour clone compared to economic agent (performance in static environment vs. structural change, policy shift)
    \item What happens at equilibrium (What if we have RL agent designing mechanisms and RL players participate the game?)
\end{itemize}
    
\end{frame}


\begin{frame}{Not so good part}
\begin{itemize}
    \item Not very robust 
\end{itemize}
    
\end{frame}
\end{document}
