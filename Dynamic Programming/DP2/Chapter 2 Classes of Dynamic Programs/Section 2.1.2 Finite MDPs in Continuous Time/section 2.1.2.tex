%%%%%%%%%%%%%%%%%%%%%%%%%%%%%%%%%%%%%%%%%
% Professional Mathematical Presentation Template
% 
% This template uses the beamer class with the Madrid theme
% and a custom color scheme for a clean, professional look
% that works well with mathematical content.
%%%%%%%%%%%%%%%%%%%%%%%%%%%%%%%%%%%%%%%%%

\documentclass[aspectratio=169]{beamer} % 16:9 aspect ratio (modern)

% Theme settings
\usetheme{Madrid}
\usecolortheme{default}


\definecolor{primcolor}{RGB}{25,50,100} % Dark blue
\setbeamercolor{structure}{fg=primcolor}
\setbeamercolor{frametitle}{bg=primcolor!15, fg=primcolor}
\setbeamercolor{title}{fg=white} % White title text for contrast
\setbeamercolor{subtitle}{fg=white} % White subtitle text
\setbeamercolor{author}{fg=primcolor} % White author text
\setbeamercolor{date}{fg=primcolor} % White date text
\setbeamercolor{institute}{fg=primcolor} % White institute text

% Font settings
\usefonttheme{professionalfonts}
\usefonttheme{serif}

% Package imports
\usepackage{amsmath, amsfonts, amssymb, amsthm} % Math packages
\usepackage{mathtools} % Enhanced math tools
\usepackage{bm} % Bold math symbols
\usepackage{graphicx} % For images
\usepackage{booktabs} % Professional tables
\usepackage{tikz} % For diagrams
\usetikzlibrary{arrows, positioning, matrix, decorations.pathreplacing}

% Use beamer's theorem styles
\setbeamertemplate{theorem}[ams style]
\setbeamertemplate{theorems}[numbered]


% Remove navigation symbols
\setbeamertemplate{navigation symbols}{}

% Custom footer
\setbeamertemplate{footline}{
  \leavevmode%
  \hbox{%
  \begin{beamercolorbox}[wd=.333333\paperwidth,ht=2.25ex,dp=1ex,center]{author in head/foot}%
    \usebeamerfont{author in head/foot}\insertshortauthor
  \end{beamercolorbox}%
  \begin{beamercolorbox}[wd=.333333\paperwidth,ht=2.25ex,dp=1ex,center]{title in head/foot}%
    \usebeamerfont{title in head/foot}\insertshorttitle
  \end{beamercolorbox}%
  \begin{beamercolorbox}[wd=.333333\paperwidth,ht=2.25ex,dp=1ex,right]{date in head/foot}%
    \usebeamerfont{date in head/foot}\insertshortdate{}\hspace*{2em}
    \insertframenumber{} / \inserttotalframenumber\hspace*{2ex} 
  \end{beamercolorbox}}%
  \vskip0pt%
}

% Title information
\title[DP2]{Dynamic Programming}
\subtitle{Thomas J. Sargent and John Stachurski}
\author[Longye]{Longye Tian \\ \texttt{longye.tian@anu.edu.au}}
\institute[ANU]{Australian National University\\ School of Economics}
\date{March 7th, 2025}
\DeclareFontFamily{U}{mathx}{\hyphenchar\font45}
\DeclareFontShape{U}{mathx}{m}{n}{
      <5> <6> <7> <8> <9> <10>
      <10.95> <12> <14.4> <17.28> <20.74> <24.88>
      mathx10
      }{}
\DeclareSymbolFont{mathx}{U}{mathx}{m}{n}
\DeclareMathSymbol{\bigtimes}{1}{mathx}{"91}

\begin{document}

% Title frame
\begin{frame}
  \titlepage
\end{frame}

% Outline frame
\begin{frame}{Outline}
  \tableofcontents
\end{frame}
\section{Primitives and Values}
\begin{frame}{Primitives and Values}
\begin{itemize}
    \item \textbf{discount rate} $\delta>0$
    \item \textbf{Intensity kernel} $Q: G\times \mathbf{X}\to \mathbb{R}$ satisfiying
    $$
    \sum_{x'} Q(x,a,x') = 0 \,\,\text{for all $(x,a)\in G$, and $Q(x,a,x')\ge 0$ when $x\neq x'$}
    $$
    \item for any $\sigma\in\Sigma$, we obtain the intensity operator or \textbf{infinitesimal generator}
    $$
    Q_\sigma(x,x'):= Q(x,\sigma(x),x')
    $$
    that determines a continuous time Markov chain $(X_t)_{t\ge 0}$ with transition probabilities give by 
    $$
    P^\sigma_t:= e^{tQ_\sigma}  \quad \text{for all $x\in\mathbf{X}$}
    $$
\end{itemize}
    
\end{frame}

\begin{frame}{Primitives and Values}
We have
$$
\mathbb{E}_x h(X_t) = (P_t^\sigma h)(x) \quad\text{for any $h\in\mathbb{R}^{\mathbf{X}}$}
$$
Hence, the lifetime value of following $\sigma$ starting from state $x$ is
$$
v_\sigma(x) = \mathbb{E}_x\int_0^\infty e^{-\delta t} r_\sigma(X_t) \, dt = \int_0^\infty e^{-\delta t}(P^\sigma_t r_\sigma)(x)\, dt
$$
Using $\delta>0$, we can write $v_\sigma$ as
$$
v_\sigma  = \int_0^\infty e^{t(Q_\sigma -\delta I)}r_\sigma\, dt = (\delta I -Q_\sigma)^{-1} r_\sigma
$$
    
\end{frame}
\begin{frame}{ADP Reformulation}
Define
$$
P(x,a,x') :=\mathbf{1}\{x=x'\} + \frac{Q(x,a,x')}{m},\quad \text{where $m:=\max_{x\in\mathbf{X}, a\in\mathbf{A}} |Q(x,a,x)|$}
$$
In addition, we set
$$
\beta:= \frac{m}{m+\delta}\quad \text{and} \quad \hat{r}_\sigma := \frac{r_\sigma}{m+\delta}
$$
\end{frame}

\begin{frame}{Exercise 2.1.9}
Prove that with the above definitions,
\begin{itemize}
    \item $P_\sigma$ is a stochastic matrix
    \item $\sigma$-value function $v_\sigma$ obeys
    $$
    v_\sigma = (I-\beta P_\sigma)^{-1} \hat{r}_\sigma 
    $$
\end{itemize}
    
\end{frame}

\begin{frame}{ADP Formulation}
We see that $v_\sigma$ is the unique fixed point in $V:=\mathbb{R}^{\mathbf{X}}$ of the policy operator
$$
T_\sigma v = \hat{r}_\sigma +\beta P_\sigma v
$$

Letting $\mathbb{T}_{CTMDP}:=\{T_\sigma:\sigma \in\Sigma\}$, we study the ADP $(\mathbb{R}^{\mathbf{X}}, \mathbb{T}_{CTMDP})$
\end{frame}


\begin{frame}{Optimality}
We can directly apply the discrete time MDP theory to get all the optimality result.
\end{frame}
\end{document}
